\documentclass[12pt,a4paper]{article}
\usepackage{amsmath}    % need for subequations
\usepackage{hyperref}   % use for hypertext links, including those to external documents and URLs
\usepackage{esint}
\usepackage{amsfonts}
\usepackage[utf8]{inputenc}
\usepackage{makeidx}
\usepackage{amsfonts}
\usepackage{fullpage}
\usepackage[spanish]{babel}
\decimalpoint

\title{Formulario de STRA - ETSIT UPM}
\author{Carlos García-Mauriño}
\date{\today}

\begin{document}

\maketitle

\twocolumn

\section{Caracterización de la señal digital}
\label{sec:caracterizacion_de_la_senal_digital}

\subsection{MIC}
\label{sub:mic}

\subsubsection{Intervalo en el segmento}
\label{ssub:intervalo_en_el_segmento}

\[ \frac{|x(n)| - \mbox{cota.inf.S }}{\Delta_S} \]

\subsubsection{Valor de recontrucción}
\label{ssub:valor_de_recontruccion}

\[ y(n) = [\mbox{signo}] \left( \mbox{cota.inf.S }
			+\frac{IC+\frac{1}{2}}{\frac{1}{\Delta_S}} \right) UTN \]

\subsubsection{Error de recontrucción}
\label{ssub:error_de_recontruccion}

\[ q(n) = | x(n) - y (n) | \]

\[ q_{max} = \frac{\Delta_s}{2} UTN \]

\section{Medios de transmisión}
\label{medios_de_transmision}

\subsection{Generales}
\label{sub:generales}


\subsubsection{Ideales}
\label{ssub:ideales}

La señal solo se verá afectada por un factor de escala y un retardo. Su función
de transferencia será:

\[ H(f) = k e^{-j \omega t_0} \]

\subsubsection{Velocidad}
\label{ssub:velocidad}

Retardo de fase y retardo de grupo:

\[ t_{fase} = \frac{\beta(f_0)}{\omega_0} d \]

\[ t_{grupo} = \frac{1}{2\pi} \frac{\partial \beta(f)}{\partial f} d \]

\subsection{Líneas metálicas}
\label{sub:lineas_metalicas}

\subsubsection{Parámetros de transmisión}
\label{ssub:parametros_de_transmision}

\[ Z_0 = \sqrt{\frac{R + j \omega L}{G + j \omega C}} \]

\[ \gamma = \sqrt{(R + j \omega L)(G + j \omega C)} \]

\subsubsection{Caso general de líneas de transmisión}
\label{ssub:caso_general_de_lineas_de_transmision}

Tensión incidente al inicio de la línea:

\[ v_i (0) = \frac{E}{2} \frac{Z(0) + Z_0}{Z(0) + Z_g} \]

Coeficiente de reflexión:

\[ \rho (d) = \frac{Z(d) - Z_0}{Z(d) + Z_0} \]

\[ \rho(0) = \rho(d) e^{-2 \gamma d} \]

Tensión:

\[ v(d) = v_i [1 + \rho (d) ] e^{- \alpha d} \]

\subsubsection{Aproximación de alta frecuencia}
\label{ssub:aproximacion_de_alta_frecuencia}

Se produce distorsión lineal (solamente de amplitud).

\[ Z_0 \approx \frac{L}{C} \quad \gamma \approx = \frac{R}{2 z_0} \quad
\beta \approx \omega \sqrt{LC} \]

\subsubsection{Línea sin distorsión}
\label{ssub:linea_sin_distorsion}

Condición de \textbf{Heaviside}: $ RC = LG $.

\subsection{Fibra óptica}
\label{sub:fibra_optica}

\subsubsection{Dispersión en las fibras monomodo}
\label{ssub:dispersion_en_las_fibras_monomodo}

Dispersión de guía-onda:

\[ G(\lambda)[ns/nm \cdot km] \approx - \frac{\lambda [nm]}{4 \pi a^2 [nm] n_1
c[km/ns]} \]

\subsubsection{Dispersión en las fibras multimodo}
\label{ssub:dispersion_en_las_fibras_multimodo}

Dispersión modal:

\[ \sigma_{mod} [ns] = 0.187 \frac{d^{\gamma}[km]}{B_0 [GHz \cdot km]} \]

\subsection{Radio}
\label{sub:radio}

\subsubsection{Pérdidas}
\label{ssub:perdidas}

\[ L_{bf} [dB] = 92.45 + 20 \log{d[km]} + 20 \log{f [GHz]} \]

\subsubsection{Antenas}
\label{ssub:antenas}

Parabólicas reales:

\[ G [dB] = 20.4 + 10 \log{k} + 20 \log{D[m]} + 20 \log{f[GHz]} \]


\section{Distorsión de la señal}
\label{distorsion_de_la_senal}

\subsection{Ruido}
\label{sub:ruido}

\[ n_0 = k t b \]

\subsubsection{Atenuadores}
\label{ssub:atenuadores}

Para atenuadores pasivos con factor de atenuación $ a $, figura de ruido $ f $,
temperatura equivalente de ruido $ t_e $, temperatura de referencia $ t_0 $ y
temperatura del atenuador $ t_{at} $.

\[ f = 1 + (a-1) \frac{t_e}{t_0} \]

\[ t_e = t_{at} (a -1) \]

\subsubsection{Amplificadores}
\label{ssub:amplificadores}

\[ f = \frac{t_e}{t_0} +1 \]

\[ t_e = t_0 (f - 1) \]

\section{Sistemas de transmisión digital}
\label{sistemas_de_transmision_digital}

\subsection{QAM}
\label{sub:qam}

\[ b = v_t (1+\alpha) \]

\[ v_t = \frac{v_b}{\log_2 M} \]

\[ P_e = 2Q \left( \sqrt{\frac{6 E_s}{N_0 (M-1)}} \right) \]
\[ = 2Q \frac{6 (1 + \alpha)}{(M-1)} \frac{s}{n} \]

\[ E_s = p_r \cdot T_s \]

\[ T_s = \frac{1}{v_t} \]

\[ v_t = \frac{R}{\log_2 M} \]

\subsection{Fibra óptica}
\label{sub:fibra_optica}

\[ \sigma^2 = \left| \frac{\lambda^2 D(\lambda)}{2 \pi c} \right| d \]

\subsubsection{Limitación por dispersión}
\label{ssub:limitacion_por_dispersion}

Condición para que $ PP $ sea menor a $ 2 dB $:

\[ \sigma \leq 0.498 T_b \]

\[ \sigma_{\lambda} = \frac{\Delta \lambda}{2.35} \]

\begin{itemize}
		\item Gran ancho espectral:
				\[ \sigma_{\lambda} \gg 1 nm \]
		\item Casi monocromáticas:
				\[ \sigma_{\lambda} < 0.001 nm \]
\end{itemize}

\subsubsection{EDFA}
\label{ssub:edfa}

\[ \eta_{ASE} = \frac{fhgv}{2} \quad \left[\frac{W}{Hz}\right] \]

\[ v = \frac{c}{\lambda} \]

\end{document}
